%%%%%%%%%%%%%%%%%%%%%%%%%%%%%%%%%%%%%%%%%
% "ModernCV" CV and Cover Letter
% LaTeX Template
% Version 1.11 (19/6/14)
%
% This template has been downloaded from:
% http://www.LaTeXTemplates.com
%
% Original author:
% Xavier Danaux (xdanaux@gmail.com)
%
% License:
% CC BY-NC-SA 3.0 (http://creativecommons.org/licenses/by-nc-sa/3.0/)
%
% Important note:
% This template requires the moderncv.cls and .sty files to be in the same 
% directory as this .tex file. These files provide the resume style and themes 
% used for structuring the document.
%
%%%%%%%%%%%%%%%%%%%%%%%%%%%%%%%%%%%%%%%%%

%----------------------------------------------------------------------------------------
%	PACKAGES AND OTHER DOCUMENT CONFIGURATIONS
%----------------------------------------------------------------------------------------

\documentclass[11pt,a4paper,liberation sans]{moderncv} % Font sizes: 10, 11, or 12; paper sizes: a4paper, letterpaper, a5paper, legalpaper, executivepaper or landscape; font families: sans or roman

\moderncvstyle{casual} % CV theme - options include: 'casual' (default), 'classic', 'oldstyle' and 'banking'
\moderncvcolor{blue} % CV color - options include: 'blue' (default), 'orange', 'green', 'red', 'purple', 'grey' and 'black'

\usepackage{lipsum} % Used for inserting dummy 'Lorem ipsum' text into the template

\usepackage[utf8]{inputenc}
\usepackage[english,greek]{babel}
\usepackage[scale=0.75]{geometry} % Reduce document margins
\setlength{\hintscolumnwidth}{3cm} % Uncomment to change the width of the dates column
%\setlength{\makecvtitlenamewidth}{10cm} % For the 'classic' style, uncomment to adjust the width of the space allocated to your name

%----------------------------------------------------------------------------------------
%	NAME AND CONTACT INFORMATION SECTION
%----------------------------------------------------------------------------------------

\firstname{Άντι} % Your first name
\familyname{Άσκο} % Your last name

% All information in this block is optional, comment out any lines you don't need
\title{Βιογραφικό Σημείωμα}
%\address{15 Kydonion St.}{Serres, Greece 62120}
%\mobile{(+30) 6980 311 101}
%\phone{(+30) 2321 059 358}
%\email{askoanti@gmail.com}
%\extrainfo{additional information}
%\homepage{antiasko.github.io}{AntiAsko.github.io} % The first argument is the url for the clickable link, the second argument is the url displayed in the template - this allows special characters to be displayed such as the tilde in this example
%\photo[70pt][0.4pt]{pictures/picture} % The first bracket is the picture height, the second is the thickness of the frame around the picture (0pt for no frame)
%\quote{"Knowledge is never enough"}

%----------------------------------------------------------------------------------------

\begin{document}

\makecvtitle % Print the CV title

%----------------------------------------------------------------------------------------
%	PERSONAL INFO SECTION
%----------------------------------------------------------------------------------------

\section{Προσωπικά Στοιχεία}

\cvitem{Ονοματεπώνυμο}{\textbf{Άντι Άσκο}}
\cvitem{Διεύθυνση}{\textbf{Κυδωνίων 15, Σέρρες}}
\cvitem{\latintext E-mail}{\latintext \textbf{\href{mailto:askoanti@gmail.com}{askoanti@gmail.com}}}
\cvitem{\latintext Website}{\latintext \href{http://antiasko.github.io}{\textbf{AntiAsko.github.io}}}
\cvitem{Κινητό}{\textbf{+30 6980 311 101}}
\cvitem{Σταθερό}{\textbf{+30 2321 059 358}}
\cvitem{Φύλο}{\textbf{Άρρεν}}
\cvitem{Ημ. Γέννησης}{\textbf{10/9/1992}}
\cvitem{Εθνικότητα}{\textbf{Ελληνική, Αλβανική}}

%----------------------------------------------------------------------------------------
%	PERSONAL STATEMENT
%----------------------------------------------------------------------------------------

%\section{Personal Statement}

%\cvitem{}{\textit{An Information Technology undergraduate from the University of Central Macedonia, Greece, I have skills and knowledge essential for contributing as }}

%----------------------------------------------------------------------------------------
%	EDUCATION SECTION
%----------------------------------------------------------------------------------------

\section{Εκπαίδευση}

\cventry{9/2010--Σήμερα}{Τμήμα Μηχανικών Πληροφορικής Τ.Ε.}{\newline{}Τεχνολογικό Εκπαιδευτικό Ίδρυμα Κεντρικής Μακεδονίας, Σέρρες}{}{}{} % Arguments not required can be left empty

\section{Πτυχιακή Εργασία (σε εξέλιξη)}
\cvitem{Τίτλος}{\emph{Ανάπτυξη εφαρμογής για την καταγραφή της μετακίνησης πληθυσμών με σκοπό την εξαγωγή πληροφοριών για πιθανές διαδρομές μετάδοσης ασθενειών. \href{https://github.com/TraceGerm}{\textbf{\latintext Website}}}}
\cvitem{}{Δρ. Νικόλαος Πεταλίδης}

%----------------------------------------------------------------------------------------
%	WORK EXPERIENCE SECTION
%----------------------------------------------------------------------------------------

\section{Επαγγελματική Εμπειρία}

\subsection{Εργασία και σπουδές}

\cventry{6/2014--8/2014}{\latintext CERN OpenLab Summer Student}{\newline{\textsc{\latintext CERN}}}{Γενεύη}{Ελβετία}{Παραβρέθηκα σε υψηλών προδιαγραφών παρουσιάσεις με θέμα διάφορους τομείς της πληροφόρικής. Ήμουν υπεύθυνος για την Εγκατάσταση, Διαχείρηση και παραμετροποίηση ενος καινούριου εργαλείου ονομαζόμενο {\latintext Oracle GoldenGate Veridata,} για τη συνοχή μεταξύ των δεδομένων στο {\latintext WLCG distributed database environment}. Επιπλέον, ήμουν υπεύθυνος για τον έλεγχο της συνοχής μεταξύ του κέντρου δεδομένων στο {\latintext CERN} και 4 ακόμα κέντρών (Αγγλία, ΗΠΑ, Γαλλία, Καναδάς). Για περισσότερες πληροφορίες πρακαλώ \href{http://openlab.web.CERN.ch/publications/technical_documents/investigation-oracle-goldengate-veridata-data-consistency-wlcg}{\textbf{πατήστε εδώ}}}

%------------------------------------------------

\cventry{9/2011--6/2014}{Τεχνικός Υπολογιστών}{\newline{\textsc{\latintext Official Internet and games}}}{Σέρρες}{Ελλάδα}{Κατά τη διάρκεια των πρώτων 3 χρόνων των σπουδών μου, συγχρώνος εργαζόμουν σε μια  τοπική επιχείρηση σαν τεχνικός υπολογιστών. Ήμουν υπεύθυνος για τη συντήρηση 25 προσωπικών υπολογιστών και 5 σέρβερ.}


%----------------------------------------------------------------------------------------
%	COMPUTER SKILLS SECTION
%----------------------------------------------------------------------------------------

\section{Γνώσεις Ηλεκτρονικών Υπολογιστών}

\cvitem{Γλώσσες \newline{}Προγραμματισμού}{\latintext Java, Javascript, SQL, PL\textbackslash SQL, (Advanced) \newline{}  Php, Shell scripting, Python (Good) \newline{} C, C++, C\#, Octave (familiar) \newline {}}

\cvitem{Βάσεις}{\latintext Oracle, MySql, MongoDB, Neo4j\newline{}}

\cvitem{Εργαλεία}{\latintext Netbeans, Eclipse, Visual Studio, Git, Intellij IDEA, Sublime, Matlab, Android SDK, \LaTeX, Joomla, Wordpress, Oracle ColdenGate Veridata, Putty etc.\newline{}}

\cvitem{\latintext Frameworks}{\latintext Sring, Hibernate, Bottle, Cordova, AngularJS, Ionic, .NET, Django, Junit, Dunit, Mockito}

%----------------------------------------------------------------------------------------
%	CERTIFICATIONS SECTION
%----------------------------------------------------------------------------------------


\section{Πιστοποιήσεις}
\cvitem{10/2014}{\textbf{\latintext M101J: MongoDB for Java Developers}\newline{} \textit{\latintext A course offered by MongoDB, Inc.}}{}
\cvitem{11/2014}{\textbf{\latintext M101P: MongoDB for Python Developers}\newline{} \textit{\latintext A course offered by MongoDB, Inc.}}{}


%----------------------------------------------------------------------------------------
%	COMMUNICATION SKILLS SECTION
%----------------------------------------------------------------------------------------

\section{Επικοινωνιακές δεξιότητες}
\cvitem{}{Με την εμπειρία μου σαν τεχνικός υπολογιστών αλλά και σαν {\latintext OpenLab student}, λόγω της καθημερινής επαφής με διάφορους ανθρώπους έχω αναπτύξει καλές επικοινωνιακές δεξιότητες. Επιπλέον σαν φοιτητής έχω πάρει μέρος σε πολλά ομαδικά πρότζεκτ τα οποία με βοήθησαν να αναπτύξω ομαδικό πνεύμα. Ολοκληρώνοντας κατα τη διάρκεια του {\latintext internship} στο {\latintext CERN} μου δόθηκε η ευκαιρία να κάνω 2 παρουσιάσεις για το πρότζεκτ μου:}
\cventry{}{\latintext Oral Presentation at the IT-DB department of CERN}{}{}{}{\textit{\latintext "Investigation on Oracle GoldenGate Veridata for Data Consistency in WLCG Distributed Database Environment"}}
\cventry{}{\latintext Lightning talk in IT group of CERN}{}{}{}{\textit{\latintext "Veridata for Data Consistency in WLCG"}}

%----------------------------------------------------------------------------------------
%	LANGUAGES SECTION
%----------------------------------------------------------------------------------------

\section{Γλώσσες}

\cvitemwithcomment{Ελληνικά}{Μητρική Γλώσσα}{}
\cvitemwithcomment{Αλβανικά}{Μητρική Γλώσσα}{}
\cvitemwithcomment{Αγγλικά}{\latintext Proficiency}{\latintext C2 Level Certificate 2008}


%----------------------------------------------------------------------------------------
%	INTERESTS SECTION
%----------------------------------------------------------------------------------------

%\section{Links}

%\renewcommand{\listitemsymbol}{~} % Changes the symbol used for lists
%\cvlistdoubleitem{Cooking}{Dancing}
%\cvlistitem{Running}
%\cvlistitem{\href{http://openlab.web.CERN.ch/publications/technical_documents/investigation-oracle-goldengate-veridata-data-consistency-wlcg}{\textbf{Github}}}

%----------------------------------------------------------------------------------------

\end{document}